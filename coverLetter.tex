%% start of file `template.tex'.
%% Copyright 2006-2013 Xavier Danaux (xdanaux@gmail.com).
%
% This work may be distributed and/or modified under the
% conditions of the LaTeX Project Public License version 1.3c,
% available at http://www.latex-project.org/lppl/.


\documentclass[10pt,a4paper,sans]{moderncv}
% possible options include font size ('10pt', '11pt' and '12pt'),
% paper size ('a4paper', 'letterpaper', 'a5paper', 'legalpaper', 'executivepaper')
% font family ('sans' and 'roman')

% moderncv themes
\moderncvstyle{banking}
% style options are 'casual' (default), 'classic', 'oldstyle' and 'banking'
\moderncvcolor{blue}
% color options 'blue' (default), 'orange', 'green', 'red', 'purple', 'grey' and 'black'
%\renewcommand{\familydefault}{\sfdefault}
%\nopagenumbers{}

% character encoding
\usepackage[utf8]{inputenc}
% if you are not using xelatex ou lualatex, replace by the encoding you are using

% adjust the page margins
\usepackage[scale=0.85]{geometry}
%\setlength{\hintscolumnwidth}{3cm}
% if you want to change the width of the column with the dates
%\setlength{\makecvtitlenamewidth}{10cm}
% for the 'classic' style, if you want to force the width allocated to your name and avoid line breaks. be careful though, the length is normally calculated to avoid any overlap with your personal info; use this at your own typographical risks...

\usepackage{ragged2e} %to justify

% personal data
\name{Jos\'e Alonso}{Sol\'is-Lemus}
%\title{Resumé title}
% optional, remove / comment the line if not wanted
\address{4, Edward Dodd Court, Chart Street}{N1 6DQ, London}{}
\phone[mobile]{+44~(0)~7838~189458}
%\phone[fixed]{+2~(345)~678~901}
\email{jose.solis-lemus.2@city.ac.uk}
\homepage{alonsojasl.github.io}
%\extrainfo{Github: alonsoJASL}
%\photo[64pt][0.4pt]{picture}
%\quote{Some quote

%\makeatletter
%\renewcommand*{\bibliographyitemlabel}{\@biblabel{\arabic{enumiv}}}
%\makeatother
%\renewcommand*{\bibliographyitemlabel}{[\arabic{enumiv}]}% CONSIDER REPLACING THE ABOVE BY THIS

% bibliography with mutiple entries
%\usepackage{multibib}
%\newcites{book,misc}{{Books},{Others}}

% university name:
\newcommand{\university}{Imperial College London}
\newcommand{\department}{Department of Computing, Faculty of Engineering}
\newcommand{\position}{Research Assistant/Associate in Biomedical Image Analysis}
\newcommand{\responsible}{Professor Daniel Rueckert}
% My info
%----------------------------------------------------------------------------------
%            content
%----------------------------------------------------------------------------------
\begin{document}
%----- letter  ---------------------------------------------------------
% recipient data
\recipient{Chair, Search Committee for \position}
{\responsible\\
  \department{} \\
  \university{}\\
}
\date{\today}
\opening{Dear Search Committee Members,}
\closing{Yours faithfully,}
%\enclosure[\textbf{Attached}]{curriculum vit\ae. \\
%\textbf{Upon request}: diversity statement, transcripts.}
\makelettertitle
\justify

Re: \position\ at the \department .

I wish to express the keen interest in this position due to its exciting challenges and positive outcomes it would have in my academic career. I have recently completed my PhD and I am currently awaiting for my viva voce examination. My PhD project consists of the development of novel image analysis algorithms of migrating cells. For my future research, I aim to focus the skills I have acquired and progress to aid in medical diagnosis, especially in prevention of diseases.

Our research balances biological phenomena, mathematical theory of image analysis and computational tractability. The project consisted in developing algorithms to automatically extract and analyse features, segment and perform tracking on the complex movement of migrating objects. Whilst special emphasis was devoted to macrophages observed with fluorescent confocal microscopy, the segmentation, tracking and feature extraction techniques developed can be extended to various other applications, not exclusively of cells.

Our research is highly interdisciplinary. The data of macrophages was acquired in close collaboration with the Randall Division of Cell & Molecular Biophysics at Kings College London. Furthermore, as part of a large collaborative effort investigating segmentation and tracking algorithms, we participated in the 2015 Cell Tracking Challenge. In close contact with the organisers and other participants, we produced a manuscript published in Nature Methods.

My programming skills have allowed me to produce robust and scalable software, adapting quickly to different programming languages. I thrive when developing new techniques and algorithms, putting emphasis in computational efficiency and numerical stability. I am able to work on different projects and coding styles. Always open to collaboration, wherever possible, I have always tried to make my research publicly available on platforms such as Github.

Throughout my PhD, I have displayed a strong interest and willingness to learn topics within image analysis beyond the range of my PhD project. For example, attending frequently to technical meetings held by the BMVA, in particular the topics of Transfer Learning, Machine learning, also participating in the Computer Vision and Modelling in Cancer. At City, University of London, I often participate in the Computer Vision Journal Club; constantly reviewing the state of the art techniques in image analysis and machine learning.

Finally, my communication skills and experience are very diverse. I have presented my work at international conferences to key researchers in my field. Specifically, I have presented at the IEEE Symposium of Biomedical Engineering, which has become one of the most important venues for biomedical research from a technical point of view. I have also developed sound writing skills both for academic settings and science reporting to common audiences. My communication skills extend to my teaching abilities, with diverse experience at A-Level, undergraduate and graduate levels; always receiving positive feedback from students.

Given the knowledge of the field acquired during my PhD, as well as the mathematical, engineering and computational skills briefly outlined in this letter; I believe I would be an ideal candidate for the position of Research Associate in Biomedical Image Analysis. In the attached documents section, a copy of my CV and a PDF version of this letter can be found. I am happy to provide more information upon request.

\makeletterclosing

\end{document}
%% end of file `template.tex'.
