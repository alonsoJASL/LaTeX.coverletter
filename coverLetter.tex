%% start of file `template.tex'.
%% Copyright 2006-2013 Xavier Danaux (xdanaux@gmail.com).
%
% This work may be distributed and/or modified under the
% conditions of the LaTeX Project Public License version 1.3c,
% available at http://www.latex-project.org/lppl/.


\documentclass[10pt,a4paper,sans]{moderncv} 
% possible options include font size ('10pt', '11pt' and '12pt'), 
% paper size ('a4paper', 'letterpaper', 'a5paper', 'legalpaper', 'executivepaper')
% font family ('sans' and 'roman')

% moderncv themes
\moderncvstyle{banking}                            
% style options are 'casual' (default), 'classic', 'oldstyle' and 'banking'
\moderncvcolor{blue}                                
% color options 'blue' (default), 'orange', 'green', 'red', 'purple', 'grey' and 'black'
%\renewcommand{\familydefault}{\sfdefault}        
%\nopagenumbers{}                                 

% character encoding
\usepackage[utf8]{inputenc}                       
% if you are not using xelatex ou lualatex, replace by the encoding you are using

% adjust the page margins
\usepackage[scale=0.85]{geometry}
%\setlength{\hintscolumnwidth}{3cm}                
% if you want to change the width of the column with the dates
%\setlength{\makecvtitlenamewidth}{10cm}           
% for the 'classic' style, if you want to force the width allocated to your name and avoid line breaks. be careful though, the length is normally calculated to avoid any overlap with your personal info; use this at your own typographical risks...

\usepackage{ragged2e} %to justify

% personal data
\name{Jos\'e Alonso}{Sol\'is-Lemus}
%\title{Resumé title}                               
% optional, remove / comment the line if not wanted
\address{4, Edward Dodd Court, Chart Street}{N1 6DQ, London}{}
\phone[mobile]{+44~(0)~7838~189458}
%\phone[fixed]{+2~(345)~678~901}
\email{jose.solis-lemus.2@city.ac.uk}
\homepage{alonsojasl.github.io} 
%\extrainfo{Github: alonsoJASL}
%\photo[64pt][0.4pt]{picture} 
%\quote{Some quote

%\makeatletter
%\renewcommand*{\bibliographyitemlabel}{\@biblabel{\arabic{enumiv}}}
%\makeatother
%\renewcommand*{\bibliographyitemlabel}{[\arabic{enumiv}]}% CONSIDER REPLACING THE ABOVE BY THIS

% bibliography with mutiple entries
%\usepackage{multibib}
%\newcites{book,misc}{{Books},{Others}}

% university name:
\newcommand{\university}{Imperial College London}
\newcommand{\department}{Department of Computing, Faculty of Engineering}
\newcommand{\position}{Research Assistant/Associate in Biomedical Image Analysis}
\newcommand{\responsible}{Professor Daniel Rueckert}
%----------------------------------------------------------------------------------
%            content
%----------------------------------------------------------------------------------
\begin{document}
%----- letter  ---------------------------------------------------------
% recipient data
\recipient{Chair, Search Committee for \position}
{\responsible\\
  \department{} \\
  \university{}\\
}
\date{\today}
\opening{Dear Search Committee Members,}
\closing{Yours faithfully,}
\enclosure[\textbf{Attached}]{curriculum vit\ae, research
  statement. \\ 
\textbf{Upon request}: diversity statement, transcripts.}          % use an optional
                                % argument to use a string other than
                                % "Enclosure", or redefine \enclname
\makelettertitle

\justify

Re: \position\ at the \department at \university .
I am writing to apply for the \position\ at the \department\ at \university .
I did my \textbf{PhD in Biomedical Engineering, in the area of Image Analysis at
City, University of London under the supervision of Senior Lecturer Constantino
Carlos Reyes-Aldasoro}. Recently, I submitted my thesis, and I am awating for my
\emph{viva} examination.

I wish to express the keen interest I have in this position due to my extensive work in biomedical image analysis. I have recently completed my PhD and are awaiting for my viva voce examination. The PhD project I developed involves the development of novel image analysis algorithms of migrating cells. For my future research, I aim to move the skills I possess closer to the patient, aiding in diagnosis of medical conditions, especially in prevention of diseases.

My current research balances biological phenomena, mathematical theory of image analysis and computational tractability. My PhD project consists of the development of algorithms to automatically extract and analyse features, segment and perform tracking on the complex movement of migrating objects. Whilst special emphasis has been devoted to macrophages observed with fluorescent confocal microscopy, the segmentation, tracking, feature extraction and analysis techniques developed can be extended to various other applications, not exclusively of cells. My research has been producing several peer-reviewed publications. Notably, I have published in Nature Methods, as part of a large collaborative effort investigating segmentation and tracking algorithms.

My research is highly interdisciplinary, and I feel comfortable in collaborating with researchers from various backgrounds and levels of experience. The data of macrophages was acquired in close collaboration with the Randall Division of Cell & Molecular Biophysics at Kings College London. Within City, University of London, I have collaborated in a project of systematic profiling analysis of videos for an upcoming startup.

All my research is publicly available on Github

I have displayed a strong interest and willingness to learn image analysis topics beyond the range of my PhD project. I attended the BMVA technical meetings on Transfer Learning, Machine learning and participated in the Computer Vision and Modelling in Cancer. I also participated in the Computer Vision Journal Club held internally at City, University of London; constantly reviewing the state of the art in broad techniques advanced image analysis and machine learning.

Finally, my communication skills and experience are very diverse. I have presented my work at international conferences to key researchers in my field. Specifically, I have presented my work at the IEEE Symposium of Biomedical Engineering, which has become one of the most important venues for biomedical research from a technical point of view. I have also developed sound writing skills both for academic settings and science reporting to common audiences. My communication skills extend to my teaching abilities, with diverse experience at A-Level, Undergraduate and Graduate levels; having always received positive feedback from students. 
% My research on statistical theory and emphasis on
% scalable software in machine learning will bond well in the
% \department{} given its interest in fundamental contributions to
% theory and algorithms for data science. 
My experience in teaching and
mentoring has roots on a deep commitment to prepare future
researchers, principles shared by the values of teaching excellence at
\university{}.

\makeletterclosing

\end{document}
%% end of file `template.tex'.
